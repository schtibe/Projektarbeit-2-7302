
\section{Miscellaneous}

\subsection{Vectors}
\label{sec:vectors}

We tried a few given vector classes, but we weren't satisfied
with their features. It lies in the nature of vectors, that when an 
operation on a vector is performed you usually get a new vector and 
the base vector is left unaltered. Apparently the other people who 
provided implementations for vectors have not been having this impression 
which made us implement our own Vectors.

\noindent So we made a generic interface for vectors of any size and
implemented a 2D vector with following operations:

\begin{itemize}
\item Addition
\item Subtraction
\item Norm
\item Cross product
\item Dot product
\item Normalize
\item Multiply by scalar
\item Rotate 
\item Get angle to x axis
\end{itemize}

All operations that do not result in a scalar return a new vector object
instead of changing the one that the operation is called on. We are aware
that this is slower in process time than to change the objects, but most
times we needed new vector objects anyway. Additionally it provides more
safety, since it's not possible to accidentally change a vector object
where it should be cloned.
\subsection{Linear combination}
\label{sec:linearCombination}

Often we had to solve problems where the borders of an area had been defined 
by vectors and the question was if a certain point is in that area or not.
This sort of problem is usually solved by a linear combination of the form:\\
\\
$ \vec{v} = \lambda * \vec{a} + \mu * \vec{b} $\\
\\
Since the solution to this problem is a linear equation system based on the 
dimension of the vectors we decided to implement this as a static method in 
the vector implementation. We omit the math here and present just the solutions 
for $\lambda$ and $\mu$\\

$ \lambda = \frac{-(b_x*v_y - b_y * v_x)}{a_x * b_y - a_y * b_x}$\\

$\mu = \frac{a_x * v_y - a_y * v_x}{a_x * b_y - a_y * b_x} $ \\

