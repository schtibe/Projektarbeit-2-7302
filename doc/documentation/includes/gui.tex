
\section{Graphical user interface}
\label{sec:gui}

The focus of our project was more on the simulation than on the graphical 
display of it hence the GUI is kept quite rudimentary. \\

\noindent We used the 2D game library Slick
\footnote{\href{http://slick.cokeandcode.com/}{http://slick.cokeandcode.com/}}
to implement the GUI. Slick relies on OpenGL to display the graphics.
To make things exchangeable, we tried to avoid heavy relations between
the GUI and the real simulation (except for the update cycle).  Therefore
for every element in the simulation that has to be displayed we made a
wrapper class that takes care of the drawing. \\

\noindent Additionally we used the Nifty Library
\footnote{\href{http://sourceforge.net/projects/nifty-gui/}
{http://sourceforge.net/projects/nifty-gui/}} which is a library to
display buttons and other GUI elements. These GUIs can easily be described
in an XML file. \\

\noindent An important thing to note is that the GUI calls the update routines of the
simulation. This is documented in more details in section \ref{sec:process}.

\subsection{Bézier Curves}

The Slick library comes with a routine to draw first grade Bézier curves.
Unfortunately this algorithm drew very edged curves which did not correspond
with the curves of the simulation. Therefore we had to draw the paths with
our own Bézier algorithm.

\subsection{Options}

We added some options to the simulation GUI:

\begin{description}

\item[FPS] Show the frame rate per second
\item[Grid] Show a grid of 100x100 units to be able to estimate distances for debugging
\item[DriverView] Toggle the display of the driver views

\end{description}
