
\section{Environment}
\label{sec:environment}

\subsection{The road model}
\label{sec:roadModel}

The road model is defined by an XML file. It contains roads and
junctions that connect together. A road defines lanes, whereof
there can be more than one even in the same direction. It also defines
a path, consisting of straight elements. The simulation then creates the 
curves between those straight elements with bézier curves (\ref{bezier}).
For every lane there is a new path generated that goes along the given path.


\subsubsection{Lanes}
\label{sec:lanes}

Lanes are composed of lane segments, which there are two types of:
straight segments and curved segments. Straight segments are fairly
easy and only consist of the start and end point. The curved segments
have some more properties to be able to create the bézier curve. 
These lane segments have to alternate, so every straight lane is connected
to a curved one and vise-versa. \\

% probably mention how the distance finding works

\subsubsection{Junctions}
\label{sec:junctions}

Junctions can have a certain type, to be able to implement different types
of rights of way. The layout of the junction, however, is by default 
determined automatically based on the roads that it is connecting. All 
lanes with the same direction are connected together. 

TODO: document how this is done with the lanes

\noindent When the junction is built, it places junction way points on the
lanes that are connecting to it. These way points not only indicate the
junctions to the drivers but also offer the possibilities of directions
to take.

\subsubsection{Bézier curves}
\label{sec:bezier}


\subsection{XML}
\label{sec:XML}

\subsection{Collision detection}

We added some collision detection to the simulation. To cars
that are close are checked for collisions periodically.

