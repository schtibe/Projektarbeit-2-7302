
\section{Environment}
\label{sec:environment}

\subsection{The road model}
\label{sec:roadModel}

The road model is defined by an XML file (\ref{sec:XML}). It contains roads and
junctions that are connected together. A road defines lanes, whereof
there can be more than one even in the same direction. It also defines
a path, consisting of straight elements. The simulation then creates the 
curves between those straight elements with bézier curves (\ref{sec:bezier}).
For every lane there is a new path generated that goes along the given path.


\subsubsection{Lanes}
\label{sec:lanes}

Lanes are composed of lane segments, which there are two types of:
straight segments and curved segments. Straight segments are fairly
easy and only consist of the start and end point. The curved segments
have some more properties to be able to create the bézier curve. 
These lane segments have to alternate, so every straight lane is connected
to a curved one and vise-versa. \\

\noindent Only the straight segments have to be specified in the XML file. 
The curved segments in between are generated automatically.

% probably mention how the distance finding works

\subsubsection{Junctions}
\label{sec:junctions}

Junctions can have a certain type, to be able to implement different types
of rights of way. The layout of the junction, however, is by default 
determined automatically based on the roads that it is connecting. All 
lanes with the same direction are connected together. \\

TODO: document how this is done with the lanes

\noindent When the junction is built, it places junction way points on the
lanes that are connecting to it. These way points not only indicate the
junctions to the drivers but also offer the possibilities of directions
to take.

\subsubsection{Bézier curves}
\label{sec:bezier}

To be able to create any type of curves in our simulation we implemented
bézier curves for both the straight and the curved lane segments. The 
straight segments are done with linear curves, the curved lane segments
with quadratic curves. The curved segments therefore have additionally
to the start and end point a vector that indicates the bend. The reason
that we implemented the straight lines as bézier curves too is to avoid
having to treat them differently.  \\

\noindent Since the curved segments are created automatically, the bend
point is calculated automatically too. To do so, the linear segments are 
considered as straights and the intersection of them is then calculated.
This intersection point is then used as the bend point of the bézier curve.
With this algorithm, some erroneous conditions in the XML file can be
detected too, for example, lane segments that are crossing each other.

\subsection{XML}
\label{sec:XML}

As mentioned above, the road model is defined by XML files. They are 
designed quite simple, since they have to be written by hand. Most work
is done automatically by the simulation builder. Figure 
\ref{fig:xmlRoad} shows an example of a road consisting of only one 
lane with two straight segments and a speed way point restricting the
speed limit to 30. Road segment coordinates are relative to the start
point of the road. \\

\resetListing
\lstset{language=XML}
\begin{figure}[H]
\begin{lstlisting}
<road id="1" startX="20" startY="0">
	<leftlanes>

	</leftlanes>
	<rightlanes>
		<lane id="1" width="1" >
			<waypoints>
				<SpeedWaypoint id="1" value="30" position="0.2" />
			</waypoints>
		</lane>
	</rightlanes>
	<roadsegments>
		<roadsegment order="1" startX="0" startY="0" endX="380" endY="0" />
		<roadsegment order="2" startX="400" startY="20" endX="400" endY="460"/>
	</roadsegments>
</road>
\end{lstlisting}
\caption{An example of a road definition in the XML file}
\label{fig:xmlRoad}
\end{figure}

Figure \ref{fig:xmlJunction} shows a junction that connects 3 roads of the type 
''CrossRoad``:

\begin{figure}[H]
\begin{lstlisting}
<junction type="CrossRoad" id ="15">
	<roads>
		<road id="21" />
		<road id="23" />
		<road id="24" />
	</roads>
</junction>
\end{lstlisting}
\caption{An example of a junction connecting 3 roads}
\label{fig:xmlJunction}
\end{figure}

The XML file is checked with an XSD file. Like this we can automatically
avoid certain errors in the XML file without having to check these
things further in the code.


